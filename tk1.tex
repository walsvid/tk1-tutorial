% Encoding: UTF8
% \documentclass{ctexart}

\documentclass[openany]{ctexbook}
\usepackage{graphicx}
\usepackage{amsfonts}
\usepackage{amsmath}
\usepackage{amssymb}
\usepackage[mathscr]{eucal}
\usepackage{pythonhighlight}
\usepackage{fancyhdr}
\usepackage{bbm}
\usepackage{textcomp}
\usepackage[top=1in,bottom=1in, left=1in, right=1in]{geometry}
\usepackage[hidelinks]{hyperref}
\usepackage{subcaption}
\usepackage{listings}
\usepackage{xcolor}
\hypersetup{
    colorlinks,
    linkcolor={red!50!black},
    citecolor={blue!50!black},
    urlcolor={blue!80!black}
}
\lstset{
    columns=fixed,
    numbers=left,                                        % 在左侧显示行号
    frame=none,                                          % 不显示背景边框
    backgroundcolor=\color[RGB]{245,245,244},            % 设定背景颜色
    keywordstyle=\color[RGB]{40,40,255},                 % 设定关键字颜色
    numberstyle=\footnotesize\color{darkgray},           % 设定行号格式
    commentstyle=\it\color[RGB]{0,96,96},                % 设置代码注释的格式
    stringstyle=\rmfamily\slshape\color[RGB]{128,0,0},   % 设置字符串格式
    showstringspaces=false,                                       % 设置语言
}

\newcommand{\warn}[1] {\fcolorbox{red!20}{red!20} {\color{red} #1}}
\newcommand{\info}[1] {\fcolorbox{blue!20}{blue!20} {\color{blue} #1}}


\begin{document}

\begin{titlepage}
  \pagenumbering{roman}
  \title{NVIDIA Jetson TK1使用介绍文档}
  \author{闻超\\v0.1.0}
  \date{\today}
  \maketitle
  \setcounter{page}{1}
  \tableofcontents
\end{titlepage}

\pagenumbering{arabic}
\chapter{NVIDIA Jetson TK1介绍}
\section{硬件参数介绍}
\paragraph{尺寸}
127mm x 127mm
\paragraph{TK1 SOC}
CPU+GPU+ISP一体设计
\paragraph{GPU}
NVIDIA Kepler ``GK20a'' GPU with 192 SM3.2 CUDA cores (upto 326 GFLOPS)
\paragraph{CPU}
NVIDIA 2.32GHz ARM quad-core Cortex-A15 CPU
\paragraph{DRAM}
2GB DDR3L 933MHz EMC x16 使用 64-bit 数据位宽
\paragraph{存储}
16GB fast eMMC 4.51
\paragraph{mini-PCIe}
此接口可用于SSD、WiFi、以太网的扩展

\begin{figure}[h]
  \centering
  \includegraphics[width=5.5in]{1.jpg}
  \caption{TK1硬件组件介绍}
\end{figure}

\section{入门资料汇总}
对于第一次使用TK1的童鞋们,可以仔细阅读以下网址提供的资料:\url{http://elinux.org/Jetson_TK1}

TK1开发板的一些资料下载网址:\url{https://developer.nvidia.com/jetson-tk1}

入门视频教学:\url{http://www.iqiyi.com/a_19rrhc0aql.html}

\section{接口相关注意事项}
\begin{enumerate}
  \item Jetson TK1只提供一个USB接口,因此若想连接鼠标、键盘等外设,可以自己购买一个USB-HUB来使用
  \item Jetson TK1提供的显示接口是HDMI,如果没有HDMI接口显示器,可以买一个HDMI转VGA或者DVI接口线即可
\end{enumerate}

\section{Jetson TK1 支持的实验框架}
TK1支持许多常见的Machine Learning、Computer Vision框架。详细列表可见\url{http://elinux.org/Jetson_TK1#Tutorials_for_developing_with_Jetson_TK1}
同时,近些年深度学习多种框架出现,TK1也在不同程度上支持这些框架,这些在上面的网址中可能没有详细说明,具体请见本文档后续的内容。


\chapter{开箱与GUI系统安装}
\section{开箱使用}
首先,连接好电源和显示器、鼠标等外设之后,按电源键进入命令行界面进行带GUI的系统的安装过程。
也就是Linux的Ubuntu发行版。可以看到,屏幕上有相关提示:
\begin{figure}[h]
  \centering
  \includegraphics[width=5.5in]{2.jpg}
  \caption{第一次使用时的命令行界面}
\end{figure}

进入安装目录
{\setmainfont{Courier New Bold}                          % 设置代码字体
\begin{lstlisting}[language=bash]
cd NVIDIA-INSTALLER
\end{lstlisting}}
运行安装脚本
{\setmainfont{Courier New Bold}                          % 设置代码字体
\begin{lstlisting}[language=bash]
sudo ./installer.sh
\end{lstlisting}}
按提示输入密码,都是\warn{ubuntu}

安装完成之后,重启机器
{\setmainfont{Courier New Bold}                          % 设置代码字体
\begin{lstlisting}[language=bash]
sudo reboot
\end{lstlisting}}

至此,开箱的第一步工作已经完成,可以看到,现在已经可以进入Ubuntu系统了。
\begin{figure}[h]
  \centering
  \includegraphics[width=5.5in]{3.jpg}
  \caption{Ubuntu系统桌面}
\end{figure}

\section{配置远程连接}
暂未完成



\chapter{CUDA的安装}
\section{下载并安装CUDA包}
Jetson TK1支持NVIDIA CUDA,不过暂时仍旧只对低版本支持较好,高版本的支持情况可以等待本文档的后续更新。
下载安装包:\url{https://developer.nvidia.com/cuda-toolkit-60}
如图所示,应该选择对应的CUDA版本,由于TK的出厂时间较早,支持的比较好的版本是6.0,而且要选择\warn{Linux for Tegra}这个版本。
\begin{figure}[h]
  \centering
  \includegraphics[width=5.5in]{4.png}
  \caption{CUDA6release界面}
\end{figure}

下载后可以见到deb包
{\setmainfont{Courier New Bold}                          % 设置代码字体
\begin{lstlisting}[language=bash]
ls
cuda-repo-l4t-r19.2_6.0-42_armhf.deb
\end{lstlisting}}

安装deb包
{\setmainfont{Courier New Bold}                          % 设置代码字体
\begin{lstlisting}[language=bash]
sudo apt-get update
sudo dpkg -i xxx.deb
\end{lstlisting}}

安装samples和toolkit
{\setmainfont{Courier New Bold}                          % 设置代码字体
\begin{lstlisting}[language=bash]
sudo apt-get install cuda-samples-6-0
sudo apt-get install cuda-toolkit-6-0
\end{lstlisting}}

设置当前用户下可以访问GPU
{\setmainfont{Courier New Bold}                          % 设置代码字体
\begin{lstlisting}[language=bash]
sudo usermod -a -G video ubuntu
\end{lstlisting}}

\warn{修改环境变量}
{\setmainfont{Courier New Bold}                          % 设置代码字体
\begin{lstlisting}[language=bash]
vim ~/.bashrc
\end{lstlisting}}

在后面加上
{\setmainfont{Courier New Bold}                          % 设置代码字体
\begin{lstlisting}[language=bash]
# Add CUDA bin and lib path
export PATH=/usr/local/cuda-6.0/bin:$PATH
export LD_LIBRARY_PATH=/usr/local/cuda-6.0/lib:$LD_LIBRARY_PATH
\end{lstlisting}}

更新bashrc
{\setmainfont{Courier New Bold}                          % 设置代码字体
\begin{lstlisting}[language=bash]
source .bashrc
\end{lstlisting}}

检查是否安装正常
{\setmainfont{Courier New Bold}                          % 设置代码字体
\begin{lstlisting}[language=bash]
nvcc -V
\end{lstlisting}}
如图所示即为安装完成。

\begin{figure}[h]
  \centering
  \includegraphics[width=4in]{5.png}
  \caption{CUDA安装完成}
\end{figure}

\pagebreak
\section{编译与运行例程}
进入cuda目录并查看当前目录中是否有samples
{\setmainfont{Courier New Bold}                          % 设置代码字体
\begin{lstlisting}[language=bash]
cd /usr/local/cuda
ls
\end{lstlisting}}
\begin{figure}[h]
  \centering
  \includegraphics[width=6in]{6.png}
  \caption{目录文件}
\end{figure}
将samples拷贝到所需要的目录
\info{cuda-install-samples-6.0.sh}是bin下面的文件,应该已经添加到PATH环境变量,不需要绝对路径即可运行,如果不能运行,请检查是否添加了环境变量。
{\setmainfont{Courier New Bold}                          % 设置代码字体
\begin{lstlisting}[language=bash]
cuda-install-samples-6.0.sh /home/ubuntu/
\end{lstlisting}}
回到主目录下查看
{\setmainfont{Courier New Bold}                          % 设置代码字体
\begin{lstlisting}[language=bash]
cd
ls
\end{lstlisting}}
如果有名为
\info{NVIDIA\_CUDA-6.0\_Samples}
的文件夹拷贝samples就完成了。
下面我们编译CUDA的samples
{\setmainfont{Courier New Bold}                          % 设置代码字体
\begin{lstlisting}[language=bash]
cd NVIDIA_CUDA-6.0_Samples
make
\end{lstlisting}}
编译完成之后我们可以在
{\setmainfont{Courier New Bold} ./bin/armv7l/linux/release/gnueabihf}
找到编译好的文件

一下是运行某些例程的截图,可以自行探索。

\begin{figure*}[h]
    \centering
        \begin{subfigure}[h!]{0.5\textwidth}
            \centering
            \includegraphics[height=1.8in]{8.png}
            \caption{sample截图1}
        \end{subfigure}%
        ~
        \begin{subfigure}[h!]{0.5\textwidth}
            \centering
            \includegraphics[height=1.8in]{9.png}
            \caption{sample截图2}
        \end{subfigure}
        \caption{例程截图}
\end{figure*}

\chapter{安装OpenCV}
\section{OpenCV下载}
首先注册NVIDIA的开发者账号\url{https://developer.nvidia.com}

下载OpenCV,下载地址:\url{https://developer.nvidia.com/linux-tegra-rel-19}

选择如下两个安装包,下载
\begin{figure}[h]
  \centering
  \includegraphics[width=6in]{10.png}
  \caption{OpenCV TK1所需优化包}
\end{figure}

下载OpenCV源码,此处下载的版本虽然不是2.4.8但是仍可以编译通过,建议尽量选择与之匹配的版本,2.4.9即可。
直接从下面的网址下载即可,如果速度较慢建议通过带有VPN的机器下载完成后拷贝到TK1中。

可以点此下载:
\href{http://downloads.sourceforge.net/project/opencvlibrary/opencv-unix/2.4.9/opencv-2.4.9.zip}{OpenCV2.4.9下载}

也可以选择如下的命令行方式:
{\setmainfont{Courier New Bold}                          % 设置代码字体
\begin{lstlisting}[language=bash]
wget http://downloads.sourceforge.net/project/opencvlibrary/\
opencv-unix/2.4.9/opencv-2.4.9.zip
\end{lstlisting}}

完成后可以看到如下的文件,表明下载完成
{\setmainfont{Courier New Bold}                          % 设置代码字体
\begin{lstlisting}[language=bash]
libopencv4tegra_2.4.8.2_armhf.deb
libopencv4tegra-dev_2.4.8.2_armhf.deb
opencv-2.4.9.zip
\end{lstlisting}}

\section{安装OpenCV依赖}
首先更新源,并且安装OpenCV的若干依赖库。需要注意的是,这些库并非必须,全部安装可以支持较多格式,请根据需要进行安装。
必须安装的用MUST标出。
{\setmainfont{Courier New Bold}                          % 设置代码字体
\begin{lstlisting}[language=bash]
sudo apt-get update
# MUST Some general development libraries
sudo apt-get install build-essential -y
sudo apt-get install make cmake cmake-curses-gui g++ git  -y
# MUST Required libraries
sudo apt-get install libgtk2.0-dev pkg-config -y
sudo apt-get libavcodec-dev libavformat-dev libswscale-dev  -y
# Python libraries
sudo apt-get install python-dev python-numpy -y
# Other dependent libraries
sudo apt-get libavutil-dev libv4l-dev libeigen3-dev -y
sudo apt-get libglew1.6-dev libtbb2 libtbb-dev libjpeg-dev -y
sudo apt-get libpng-dev libtiff-dev libjasper-dev libdc1394-22-dev -y
\end{lstlisting}}

接下来安装之前下载的两个deb包
{\setmainfont{Courier New Bold}                          % 设置代码字体
\begin{lstlisting}[language=bash]
sudo dpkg -i libopencv4tegra_2.4.8.2_armhf.deb
sudo dpkg -i libopencv4tegra-dev_2.4.8.2_armhf.deb
\end{lstlisting}}

\section{源码编译OpenCV}
首先解压缩OpenCV的源代码,然后建立编译需要的build目录。
{\setmainfont{Courier New Bold}                          % 设置代码字体
\begin{lstlisting}[language=bash]
unzip opencv-2.4.9.zip
cd ~/opencv-2.4.9
mkdir build
cd build
\end{lstlisting}}
接下来我们可以查看cmake的编译选项:
{\setmainfont{Courier New Bold}                          % 设置代码字体
\begin{lstlisting}[language=bash]
ccmake ..
\end{lstlisting}}
需要注意的是,查看是否有CUDA支持,是否有Python支持。
\begin{figure}[h]
  \centering
  \includegraphics[width=5.5in]{11.png}
  \caption{ccmake编译选项的可视化显示}
\end{figure}

编译并安装OpenCV
{\setmainfont{Courier New Bold}                          % 设置代码字体
\begin{lstlisting}[language=bash]
sudo make -j4
sudo make install
\end{lstlisting}}

若能如图所示,表示OpenCV的Python支持安装成功。
\begin{figure}[h]
  \centering
  \includegraphics[width=5.5in]{12.png}
  \caption{Python显示OpenCV版本}
\end{figure}

如果想测试GPU支持,可以到如下目录运行samples
{\setmainfont{Courier New Bold}                          % 设置代码字体
\begin{lstlisting}[language=bash]
cd ~/opencv-2.4.9/samples/gpu
./hog --video 768x576.avi
\end{lstlisting}}

至此,OpenCV安装成功!

\chapter{YOLO Draknet编译}
\section{编译安装Darknet}
首先拉取repo,然后进行编译

\warn{此处的编译仅针对CPU版本,支持CUDA的GPU编译方式注意下一小节内容}
{\setmainfont{Courier New Bold}                          % 设置代码字体
\begin{lstlisting}[language=bash]
git clone https://github.com/pjreddie/darknet.git
cd darknet
make
\end{lstlisting}}
等待编译完成之后尝试运行
{\setmainfont{Courier New Bold}                          % 设置代码字体
\begin{lstlisting}[language=bash]
./darknet
\end{lstlisting}}
应该得到如下的输出
{\setmainfont{Courier New Bold}                          % 设置代码字体
\begin{lstlisting}[language=bash]
usage: ./darknet <function>
\end{lstlisting}}
至此,YOLO Darknet编译安装完成。
\section{支持CUDA的编译}
尽管YOLO Darknet已经有较快的物体检测速度,但是使用GPU可以得到数十倍甚至近百倍的速度加成,下面介绍如何在编译时支持CUDA。
需要修改Makefile文件,有以下几点需要注意。

将原来的Makefile
{\setmainfont{Courier New Bold}                          % 设置代码字体
\begin{lstlisting}[language=make]
GPU=0
...
OPENCV=0
...
ARCH= -gencode arch=compute_20,code=[sm_20,sm_21] \
      -gencode arch=compute_30,code=sm_30 \
      -gencode arch=compute_35,code=sm_35 \
      -gencode arch=compute_50,code=[sm_50,compute_50] \
      -gencode arch=compute_52,code=[sm_52,compute_52]
...
LDFLAGS+= -L/usr/local/cuda/lib64 -lcuda -lcudart -lcublas -lcurand
\end{lstlisting}}
改为支持CUDA的编译选项
{\setmainfont{Courier New Bold}                          % 设置代码字体
\begin{lstlisting}[language=make]
GPU=1
...
OPENCV=1
...
ARCH=  -gencode arch=compute_20,code=compute_20
...
LDFLAGS+= -L/usr/local/cuda/lib -lcuda -lcudart -lcublas -lcurand
\end{lstlisting}}

\section{测试安装结果}
测试一下GPU的支持以及OpenCV的支持
{\setmainfont{Courier New Bold}                          % 设置代码字体
\begin{lstlisting}[language=bash]
# Test OpenCV
./darknet imtest data/eagle.jpg
# Test GPU
wget http://pjreddie.com/media/files/yolo.weights
./darknet -i 0 detect cfg/yolo.cfg yolo.weights data/dog.jpg
\end{lstlisting}}

如下图可以认为Draknet也安装成功。

\begin{figure*}[h]
    \centering
        \begin{subfigure}[h!]{0.5\textwidth}
            \centering
            \includegraphics[height=1.8in]{13.png}
            \caption{Darknet OpenCV测试}
        \end{subfigure}%
        ~
        \begin{subfigure}[h!]{0.5\textwidth}
            \centering
            \includegraphics[height=1.8in]{14.png}
            \caption{Darknet GPU测试}
        \end{subfigure}
        \caption{Darknet运行结果截图}
\end{figure*}

\chapter{NVIDIA tk1 实际工程项目实践}
\section{使用YOLO并结合摄像头实现image/video object detection}
YOLO算法使用的darknet模型框架只依赖CUDA,可以直接在tk1上运行。
对于网络权重的训练部分在服务器端进行,tk1上直接进行测试。
对于源码分析一下:
{\setmainfont{Courier New Bold}                          % 设置代码字体
\begin{lstlisting}[language=c]
void run_yolo(int argc, char **argv)
{
    char *prefix = find_char_arg(argc, argv, "-prefix", 0);
    float thresh = find_float_arg(argc, argv, "-thresh", .2);
    int cam_index = find_int_arg(argc, argv, "-c", 0);
    int frame_skip = find_int_arg(argc, argv, "-s", 0);
    if(argc < 4){
        fprintf(stderr, "usage: %s %s [train/test/valid] \
        [cfg] [weights (optional)]\n", argv[0], argv[1]);
        return;
    }

    char *cfg = argv[3];
    char *weights = (argc > 4) ? argv[4] : 0;
    char *filename = (argc > 5) ? argv[5]: 0;
    if(0==strcmp(argv[2], "test"))
        test_yolo(cfg, weights, filename, thresh);
    else if(0==strcmp(argv[2], "train"))
        train_yolo(cfg, weights);
    else if(0==strcmp(argv[2], "valid"))
        validate_yolo(cfg, weights);
    else if(0==strcmp(argv[2], "recall"))
        validate_yolo_recall(cfg, weights);
    else if(0==strcmp(argv[2], "demo"))
        demo(cfg, weights, thresh, cam_index, \
          filename, voc_names, 20, frame_skip, prefix);
}
\end{lstlisting}}
通过分析上面的代码,在运行YOLO的时候,如果是图片,就应该选择test作为参数;
如果是视频的格式,就选择demo作为视频的参数,并且要注意传入摄像头的编号。
\subsection{yolo模型介绍}
% 这里简单介绍YOLO
\subsubsection{模型思想}
YOLO的核心思想就是利用整张图作为网络的输入,直接在输出层回归bounding box的位置和bounding box所属的类别。

faster RCNN中也直接用整张图作为输入,但是faster-RCNN整体还是采用了RCNN那种 proposal+classifier的思想,只不过是将提取proposal的步骤放在CNN中实现了。

具体采用的实现方法:
将一幅图像分成SxS个网格(grid cell),如果某个object的中心 落在这个网格中,则这个网格就负责预测这个object。
每个网格要预测B个bounding box,每个bounding box除了要回归自身的位置之外,还要附带预测一个confidence值。
这个confidence代表了所预测的box中含有object的置信度和这个box预测的有多准两重信息,其值是这样计算的:
$$Pr(Object)*IOU^{truth}_{predict}$$
其中如果有object落在一个grid cell里,第一项取1,否则取0。 第二项是预测的bounding box和实际的groundtruth之间的IoU值。

每个bounding box要预测(x, y, w, h)和confidence共5个值,每个网格还要预测一个类别信息,记为C类。则SxS个网格,每个网格要预测B个bounding box还要预测C个categories。输出就是S x S x (5*B+C)的一个tensor。
注意:class信息是针对每个网格的,confidence信息是针对每个bounding box的。

在PASCAL VOC数据集中,图像输入为448x448,取S=7,B=2,一共有20个类别(C=20)。则输出就是7x7x30的一个tensor。

在test的时候,每个网格预测的class信息和bounding box预测的confidence信息相乘,就得到每个bounding box的class-specific confidence score:
$$Pr(Class_i|Object)*Pr(Object)*IOU^{truth}_{predict}=Pr(Class_i)*IOU^{truth}_{predict}$$
等式左边第一项就是每个网格预测的类别信息,第二三项就是每个bounding box预测的confidence。这个乘积即encode了预测的box属于某一类的概率,也有该box准确度的信息。

得到每个box的class-specific confidence score以后,设置阈值,滤掉得分低的boxes,对保留的boxes进行NMS处理,就得到最终的检测结果。
\subsubsection{损失函数的设计}
在实现中,最主要的就是怎么设计损失函数,让这个三个方面得到很好的平衡。作者简单粗暴的全部采用了sum-squared error loss来做这件事。

这种做法存在以下几个问题:
\begin{enumerate}
  \item 8维的localization error和20维的classification error同等重要显然是不合理的;
  \item 如果一个网格中没有object(一幅图中这种网格很多),那么就会将这些网格中的box的confidence push到0,相比于较少的有object的网格,这种做法是overpowering的,这会导致网络不稳定甚至发散。
\end{enumerate}
解决办法:

更重视8维的坐标预测,给这些损失前面赋予更大的loss weight, 记为在pascal VOC训练中取5。
对没有object的box的confidence loss,赋予小的loss weight,记为在pascal VOC训练中取0.5。
有object的box的confidence loss和类别的loss的loss weight正常取1。
对不同大小的box预测中,相比于大box预测偏一点,小box预测偏一点肯定更不能被忍受的。而sum-square error loss中对同样的偏移loss是一样。
为了缓和这个问题,作者用了一个比较取巧的办法,就是将box的width和height取平方根代替原本的height和width。

最后的损失函数:
\begin{multline}
  \lambda_{coord}\sum_{i=0}^{S^2}\sum_{j=0}^{B}\mathbbm{1}^{obj}_{ij}[(x_i-\hat{x_i})^{2}+(y_i-\hat{y_i})^{2}]\\
  +\lambda_{coord}\sum_{i=0}^{S^2}\sum_{j=0}^{B}\mathbbm{1}^{obj}_{ij}[(\sqrt{w_i}-\sqrt{\hat{w_i}})^{2}\\
  +(\sqrt{h_i}-\sqrt{\hat{h_i}})^{2}]+\sum_{i=0}^{S^2}\sum_{j=0}^{B}\mathbbm{1}^{obj}_{ij}(C_i-\hat{C_i})^{2}\\
  +\lambda_{noobj}\sum_{i=0}^{S^2}\sum_{j=0}^{B}\mathbbm{1}^{noobj}_{ij}(C_i-\hat{C_i})^{2}+\sum_{i=0}^{S^2}\mathbbm{1}^{obj}_{i}\sum_{c \in classes}(p_i(c)-\hat{p_i}(c))^2
  \nonumber
\end{multline}
\subsubsection{模型缺点}
\begin{enumerate}
  \item YOLO对相互靠的很近的物体,还有很小的群体 检测效果不好,这是因为一个网格中只预测了两个框,并且只属于一类。
  \item 对测试图像中,同一类物体出现的新的不常见的长宽比和其他情况是。泛化能力偏弱。
  \item 由于损失函数的问题,定位误差是影响检测效果的主要原因。尤其是大小物体的处理上,还有待加强。
\end{enumerate}

\subsubsection{网络结构}
下面是YOLO算法使用的网络模型的结构:
\begin{figure}[h]
  \centering
  \includegraphics[width=5.5in]{15.png}
  \caption{YOLO的网络结构模型}
\end{figure}
{\setmainfont{Courier New Bold}                          % 设置代码字体
\begin{lstlisting}[language=bash]
layer     filters    size              input                output
    0 conv     32  3 x 3 / 1   416 x 416 x   3   ->   416 x 416 x  32
    1 max          2 x 2 / 2   416 x 416 x  32   ->   208 x 208 x  32
    2 conv     64  3 x 3 / 1   208 x 208 x  32   ->   208 x 208 x  64
    3 max          2 x 2 / 2   208 x 208 x  64   ->   104 x 104 x  64
    4 conv    128  3 x 3 / 1   104 x 104 x  64   ->   104 x 104 x 128
    5 conv     64  1 x 1 / 1   104 x 104 x 128   ->   104 x 104 x  64
    6 conv    128  3 x 3 / 1   104 x 104 x  64   ->   104 x 104 x 128
    7 max          2 x 2 / 2   104 x 104 x 128   ->    52 x  52 x 128
    8 conv    256  3 x 3 / 1    52 x  52 x 128   ->    52 x  52 x 256
    9 conv    128  1 x 1 / 1    52 x  52 x 256   ->    52 x  52 x 128
   10 conv    256  3 x 3 / 1    52 x  52 x 128   ->    52 x  52 x 256
   11 max          2 x 2 / 2    52 x  52 x 256   ->    26 x  26 x 256
   12 conv    512  3 x 3 / 1    26 x  26 x 256   ->    26 x  26 x 512
   13 conv    256  1 x 1 / 1    26 x  26 x 512   ->    26 x  26 x 256
   14 conv    512  3 x 3 / 1    26 x  26 x 256   ->    26 x  26 x 512
   15 conv    256  1 x 1 / 1    26 x  26 x 512   ->    26 x  26 x 256
   16 conv    512  3 x 3 / 1    26 x  26 x 256   ->    26 x  26 x 512
   17 max          2 x 2 / 2    26 x  26 x 512   ->    13 x  13 x 512
   18 conv   1024  3 x 3 / 1    13 x  13 x 512   ->    13 x  13 x1024
   19 conv    512  1 x 1 / 1    13 x  13 x1024   ->    13 x  13 x 512
   20 conv   1024  3 x 3 / 1    13 x  13 x 512   ->    13 x  13 x1024
   21 conv    512  1 x 1 / 1    13 x  13 x1024   ->    13 x  13 x 512
   22 conv   1024  3 x 3 / 1    13 x  13 x 512   ->    13 x  13 x1024
   23 conv   1024  3 x 3 / 1    13 x  13 x1024   ->    13 x  13 x1024
   24 conv   1024  3 x 3 / 1    13 x  13 x1024   ->    13 x  13 x1024
   25 route  16
   26 reorg              / 2    26 x  26 x 512   ->    13 x  13 x2048
   27 route  26 24
   28 conv   1024  3 x 3 / 1    13 x  13 x3072   ->    13 x  13 x1024
   29 conv    425  1 x 1 / 1    13 x  13 x1024   ->    13 x  13 x 425
   30 detection
\end{lstlisting}}

\subsection{可选的tiny-yolo模型}
由于YOLO模型相对于tk1的计算能力来说还是比较庞大,所以可以选用更小但是速度更快的tiny-YOLO来实现video级别的物体检测。
当然这样做的代价是牺牲了准确度。
% 这里展示tiny yolo
\subsection{Image object detection}
对于图片,参数选择是test,调用\info{test\_yolo}
函数。对于图片测试,只需要运行下面的命令即可。
{\setmainfont{Courier New Bold}                          % 设置代码字体
\begin{lstlisting}[language=bash]
./darknet detector test cfg/coco.data cfg/yolo.cfg yolo.weights <image path>
\end{lstlisting}}
\subsection{Video object detection}
对于视频,实时进行object detection,需要调用的是
\info{demo}
函数。需要传入待分析的视频文件或者camera的number。

tk1并不自带摄像头,好在可以使用USB摄像头,无需任何驱动,直接可用。传入的num是0。
{\setmainfont{Courier New Bold}                          % 设置代码字体
\begin{lstlisting}[language=bash]
./darknet detector demo cfg/coco.data cfg/yolo.cfg yolo.weights -c <num>
\end{lstlisting}}
或者可以传入一个视频文件,YOLO会将object detection的结果实时输出。
{\setmainfont{Courier New Bold}                          % 设置代码字体
\begin{lstlisting}[language=bash]
./darknet detector demo cfg/coco.data cfg/yolo.cfg yolo.weights <video file>
\end{lstlisting}}
\section{基于OpenCV和HOG特征的行人检测}
\subsection{HOG检测原理}
HOG特征描述子(HOG descriptors)是由Navneet Dalal和 Bill Triggs在2005年的一篇介绍行人检测方法的论文提到的特征描述子。

HOG属于特征提取,它统计梯度直方图特征。具体来说就是将梯度方向(0->360°)划分为9个区间,将图像化为16x16的若干个block,每个block在化为4个cell(8x8)。对每一个cell,算出每一点的梯度方向和模,按梯度方向增加对应bin的值,最终综合N个cell的梯度直方图形成一个高维描述子向量。实际实现的时候会有各种插值。
选用的分类器是经典的SVM。检测框架为经典的滑动窗口法,即在位置空间和尺度空间遍历搜索检测。
\subsection{HOG检测代码}
利用安装好的OpenCV2,我们调用其中的api,写一个用于行人图片检测程序。
从命令行读入图片的路径,使用HOG来进行特征检测,最后输出行人位置的bounding box。
{\setmainfont{Courier New Bold}                          % 设置代码字体
\begin{lstlisting}[language=python]
from __future__ import print_function

import numpy as np
import cv2


def inside(r, q):
    rx, ry, rw, rh = r
    qx, qy, qw, qh = q
    return rx > qx and ry > qy and rx + rw < qx + qw and ry + rh < qy + qh


def draw_detections(img, rects, thickness = 1):
    for x, y, w, h in rects:
        # the HOG detector returns slightly larger rectangles
        # than the real objects.
        # so we slightly shrink the rectangles to get a nicer output.
        pad_w, pad_h = int(0.15*w), int(0.05*h)
        cv2.rectangle(img, (x+pad_w, y+pad_h),
                    (x+w-pad_w, y+h-pad_h), (0, 255, 0), thickness)


if __name__ == '__main__':
    import sys
    from glob import glob
    import itertools as it

    print(__doc__)

    hog = cv2.HOGDescriptor()
    hog.setSVMDetector( cv2.HOGDescriptor_getDefaultPeopleDetector() )

    default = ['./data/1.jpg '] if len(sys.argv[1:]) == 0 else []

    for fn in it.chain(*map(glob, default + sys.argv[1:])):
        print(fn, ' - ',)
        try:
            img = cv2.imread(fn)
            if img is None:
                print('Failed to load image file:', fn)
                continue
        except:
            print('loading error')
            continue

        found, w = hog.detectMultiScale(img, winStride=(8,8),
                      padding=(32,32), scale=1.05)
        found_filtered = []
        for ri, r in enumerate(found):
            for qi, q in enumerate(found):
                if ri != qi and inside(r, q):
                    break
            else:
                found_filtered.append(r)
        draw_detections(img, found)
        draw_detections(img, found_filtered, 3)
        print('%d (%d) found' % (len(found_filtered), len(found)))
        cv2.imshow('img', img)
        ch = cv2.waitKey()
        if ch == 27:
            break
    cv2.destroyAllWindows()
\end{lstlisting}}
\subsection{行人检测的运行于结果评测}
\begin{figure*}[h]
    \centering
        \begin{subfigure}[h!]{0.5\textwidth}
            \centering
            \includegraphics[height=2in]{16.png}
            \caption{测试图1}
        \end{subfigure}%
        ~
        \begin{subfigure}[h!]{0.5\textwidth}
            \centering
            \includegraphics[height=2in]{17.png}
            \caption{测试图2}
        \end{subfigure}
        \caption{OpenCV行人检测运行结果截图}
\end{figure*}
通过上述的图片的测试结果,在tk1上运行的基于HOG特征利用OpenCV框架的行人检测的效果较好,能够对图片中的行人进行检测。
但是也应当看出如果是背影的行人的检测效果不佳,对于过大或者过小的目标的检测效果也有待提高。
对于HOG这种基于图像特征提取的shallow模型来说达到这种效果已经是比较令人满意了。
由于OpenCV封装的HOG特征提取的效率较高,在tk1这种嵌入式设备上也有较好的效果。
\section{基于OpenCV的 face detection}
OpenCV框架的人脸检测是基于Haar特征提取的。这里我们先暂时不分析Haar特征提取的原理,单纯看一下如何利用OpenCV进行人脸的检测。
\subsection{人脸检测的代码}
{\setmainfont{Courier New Bold}                          % 设置代码字体
\begin{lstlisting}[language=python]
#!/usr/bin/python

import sys
import cv2.cv as cv
from optparse import OptionParser

min_size = (20, 20)
image_scale = 2
haar_scale = 1.2
min_neighbors = 2
haar_flags = 0

def detect_and_draw(img, cascade):
    # allocate temporary images
    gray = cv.CreateImage((img.width,img.height), 8, 1)
    small_img = cv.CreateImage((cv.Round(img.width / image_scale),
                   cv.Round (img.height / image_scale)), 8, 1)

    # convert color input image to grayscale
    cv.CvtColor(img, gray, cv.CV_BGR2GRAY)

    # scale input image for faster processing
    cv.Resize(gray, small_img, cv.CV_INTER_LINEAR)

    cv.EqualizeHist(small_img, small_img)

    if(cascade):
        t = cv.GetTickCount()
        faces = cv.HaarDetectObjects(small_img, cascade, cv.CreateMemStorage(0),
                                     haar_scale, min_neighbors, haar_flags, min_size)
        t = cv.GetTickCount() - t
        print "detection time = %gms" % (t/(cv.GetTickFrequency()*1000.))
        if faces:
            for ((x, y, w, h), n) in faces:
                # the input to cv.HaarDetectObjects was resized, so scale the
                # bounding box of each face and convert it to two CvPoints
                pt1 = (int(x * image_scale), int(y * image_scale))
                pt2 = (int((x + w) * image_scale), int((y + h) * image_scale))
                cv.Rectangle(img, pt1, pt2, cv.RGB(255, 0, 0), 3, 8, 0)

    cv.ShowImage("result", img)

if __name__ == '__main__':

    parser = OptionParser(usage = "usage: %prog [options] \
              [filename|camera_index]")
    parser.add_option("-c", "--cascade", action="store", dest="cascade",
                    type="str", help="Haar cascade file, default %default",
                    default = "../data/haarcascades/haarcascade_frontalface_alt.xml")
    (options, args) = parser.parse_args()

    cascade = cv.Load(options.cascade)

    if len(args) != 1:
        parser.print_help()
        sys.exit(1)

    input_name = args[0]
    if input_name.isdigit():
        capture = cv.CreateCameraCapture(int(input_name))
    else:
        capture = None

    cv.NamedWindow("result", 1)

    if capture:
        frame_copy = None
        while True:
            frame = cv.QueryFrame(capture)
            if not frame:
                cv.WaitKey(0)
                break
            if not frame_copy:
                frame_copy = cv.CreateImage((frame.width,frame.height),
                                            cv.IPL_DEPTH_8U, frame.nChannels)
            if frame.origin == cv.IPL_ORIGIN_TL:
                cv.Copy(frame, frame_copy)
            else:
                cv.Flip(frame, frame_copy, 0)

            detect_and_draw(frame_copy, cascade)

            if cv.WaitKey(10) >= 0:
                break
    else:
        image = cv.LoadImage(input_name, 1)
        detect_and_draw(image, cascade)
        cv.WaitKey(0)

    cv.DestroyWindow("result")
\end{lstlisting}}
\subsection{人脸检测的结果}
\begin{figure}[h]
  \centering
  \includegraphics[width=5.5in]{18.png}
  \caption{基于Haar特征的人脸检测}
\end{figure}
检测的结果还是比较好的,但是基于Haar的特征提取器对于较小和角度不太好的人脸的检测效果一般,有待提升。
\end{document}

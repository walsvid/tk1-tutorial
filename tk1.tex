% Encoding: UTF8
% \documentclass{ctexart}

\documentclass[openany]{ctexbook}
\usepackage{graphicx}
\usepackage{amsfonts}
\usepackage{amsmath}
\usepackage{amssymb}
\usepackage[mathscr]{eucal}
\usepackage{pythonhighlight}
\usepackage{fancyhdr}
\usepackage{textcomp}
\usepackage[top=1in,bottom=1in, left=1in, right=1in]{geometry}
\usepackage[hidelinks]{hyperref}

\usepackage{listings}
\usepackage{xcolor}
\lstset{
    columns=fixed,
    numbers=left,                                        % 在左侧显示行号
    frame=none,                                          % 不显示背景边框
    backgroundcolor=\color[RGB]{245,245,244},            % 设定背景颜色
    keywordstyle=\color[RGB]{40,40,255},                 % 设定关键字颜色
    numberstyle=\footnotesize\color{darkgray},           % 设定行号格式
    commentstyle=\it\color[RGB]{0,96,96},                % 设置代码注释的格式
    stringstyle=\rmfamily\slshape\color[RGB]{128,0,0},   % 设置字符串格式
    showstringspaces=false,                                       % 设置语言
}

\newcommand{\warn}[1] {\fcolorbox{red!20}{red!20} {\color{red} #1}}
\newcommand{\info}[1] {\fcolorbox{blue!20}{blue!20} {\color{blue} #1}}


\begin{document}

\begin{titlepage}
  \pagenumbering{roman}
  \title{NVIDIA Jetson TK1使用介绍文档}
  \author{闻超\\v0.1.0}
  \date{\today}
  \maketitle
  \setcounter{page}{1}
  \tableofcontents
\end{titlepage}

\pagenumbering{arabic}
\chapter{NVIDIA Jetson TK1介绍}
\section{硬件参数介绍}
\paragraph{尺寸}
127mm x 127mm
\paragraph{TK1 SOC}
CPU+GPU+ISP一体设计
\paragraph{GPU}
NVIDIA Kepler ``GK20a'' GPU with 192 SM3.2 CUDA cores (upto 326 GFLOPS)
\paragraph{CPU}
NVIDIA 2.32GHz ARM quad-core Cortex-A15 CPU
\paragraph{DRAM}
2GB DDR3L 933MHz EMC x16 使用 64-bit 数据位宽
\paragraph{存储}
16GB fast eMMC 4.51
\paragraph{mini-PCIe}
此接口可用于SSD、WiFi、以太网的扩展

\begin{figure}[h]
  \centering
  \includegraphics[width=5.5in]{1.jpg}
  \caption{TK1硬件组件介绍}
\end{figure}

\section{入门资料汇总}
对于第一次使用TK1的童鞋们,可以仔细阅读以下网址提供的资料:\url{http://elinux.org/Jetson_TK1}

TK1开发板的一些资料下载网址:\url{https://developer.nvidia.com/jetson-tk1}

入门视频教学:\url{http://www.iqiyi.com/a_19rrhc0aql.html}

\section{接口相关注意事项}
\begin{enumerate}
  \item Jetson TK1只提供一个USB接口,因此若想连接鼠标、键盘等外设,可以自己购买一个USB-HUB来使用
  \item Jetson TK1提供的显示接口是HDMI,如果没有HDMI接口显示器,可以买一个HDMI转VGA或者DVI接口线即可
\end{enumerate}

\section{Jetson TK1 支持的实验框架}
TK1支持许多常见的Machine Learning、Computer Vision框架。详细列表可见\url{http://elinux.org/Jetson_TK1#Tutorials_for_developing_with_Jetson_TK1}
同时,近些年深度学习多种框架出现,TK1也在不同程度上支持这些框架,这些在上面的网址中可能没有详细说明,具体请见本文档后续的内容。


\chapter{开箱与GUI系统安装}
\section{开箱使用}
首先,连接好电源和显示器、鼠标等外设之后,按电源键进入命令行界面进行带GUI的系统的安装过程。
也就是Linux的Ubuntu发行版。可以看到,屏幕上有相关提示:
\begin{figure}[h]
  \centering
  \includegraphics[width=5.5in]{2.jpg}
  \caption{第一次使用时的命令行界面}
\end{figure}

进入安装目录
{\setmainfont{Courier New Bold}                          % 设置代码字体
\begin{lstlisting}[language=bash]
ubuntu@tegra-ubuntu:~\$ cd NVIDIA-INSTALLER
\end{lstlisting}}
运行安装脚本
{\setmainfont{Courier New Bold}                          % 设置代码字体
\begin{lstlisting}[language=bash]
ubuntu@tegra-ubuntu:~\$ sudo ./installer.sh
\end{lstlisting}}
按提示输入密码,都是\warn{ubuntu}

安装完成之后,重启机器
{\setmainfont{Courier New Bold}                          % 设置代码字体
\begin{lstlisting}[language=bash]
ubuntu@tegra-ubuntu:~\$ sudo reboot
\end{lstlisting}}

至此,开箱的第一步工作已经完成,可以看到,现在已经可以进入Ubuntu系统了。
\begin{figure}[h]
  \centering
  \includegraphics[width=5.5in]{3.jpg}
  \caption{Ubuntu系统桌面}
\end{figure}

\section{配置远程连接}
暂未完成



\chapter{CUDA的安装}
\chapter{安装OpenCV}
\chapter{YOLO Draknet编译}


\end{document}
